\documentclass[a4paper,12pt]{article} % My Personal Preference

\usepackage[table,xcdraw]{xcolor} % Prettier Tables
\usepackage[utf8]{inputenc} % Makes input utf8 (I think?)
\usepackage{amsmath} % All maths related things
\usepackage{amssymb} % Maths symbols

\usepackage{graphicx} % Use images in your report
\usepackage{fancyhdr} % Make very nice looking header and footers for your report
\usepackage{titlesec} % Change the way the default titles/heading look
\usepackage{float} % Make your pictures float/set where you want them
\usepackage{appendix} % Nicer appendices - acts as a section, own environment, etc.
\usepackage{siunitx} % Use SI Units in a readable manner
\usepackage{pgfplots} % Add graphs very simply
\usepackage{caption} % Options for captions
\usepackage[american]{babel} % Does something to do with formatting based on culture
\usepackage{pdflscape} % Make landscape pages display in a PDF in landscape, instead of sideways portait
\usepackage{geometry} % ability to change the sizes of a page.  usefull for `fullbleed' pages
\usepackage{mathrsfs} % Used for Maths-based script letters (like the R for the real number set)
\usepackage{listings} % format your code for pretty display
\usepackage[url=true,backend=biber,style=ieee]{biblatex} % better bibliographies
\usepackage{nameref} % Reference names and make them links (I think?)
\usepackage[backref=true]{hyperref} % Should be last package include.  Adds links that work


\usepackage{mathtools} % Math tools
\DeclarePairedDelimiter\ceil{\lceil}{\rceil} % Ceiling pattern
\DeclarePairedDelimiter\floor{\lfloor}{\rfloor}

% `caption' setup
\captionsetup{justification=centering}

% `listings' setup.  See package docs for info
\lstset{language=PHP,
	numbers=left,
	basicstyle=\ttfamily,
	keywordstyle=\color{blue}\ttfamily,
	stringstyle=\color{red}\ttfamily,
	commentstyle=\color{green}\ttfamily,
	morecomment=[l][\color{magenta}]{\#},
	literate={\ \ }{{\ }}1,
	showstringspaces=false,
	breaklines=true,
	tabsize=2
}
% Define a second listing style
\lstset{language=SQL, basicstyle=\ttfamily, keywordstyle=\color{blue}\ttfamily, stringstyle=\color{magenta}\ttfamily}

% Add in your bibliography (all references should go in here)
\addbibresource{bib.bib}

% You can add another if you want

%Fancy Header Formatting
% This is pretty much my standard, but I move the parts around based on
% how long titles are
\pagestyle{fancy}
\fancyhf{}
\setlength{\headheight}{15pt}
\lhead{E. Moore \& J. Miles}
\rhead{CAB230 - Web Computing}
\lfoot{Part \thesection}
\cfoot{Routr}
\rfoot{\thepage}
\renewcommand{\footrulewidth}{0.4pt}

\newcommand\Tstrut{\rule{0pt}{2.6ex}}         % = `top' strut
\newcommand\Bstrut{\rule[-0.9ex]{0pt}{0pt}}   % = `bottom' strut

% Change the name from Contents to Table of Contents
\addto\captionsamerican{% 
	\renewcommand{\contentsname}%
	{Table of Contents}%
}

% the title of your report - not really used.
\title{A Sample Report}

\author{Your Name}
\begin{document}
	%--------------PREAMBLE-------------------------
	% includes the title page, gives roman numerals to all pages before the first page,
	\include{titlepage}
	\pagenumbering{roman}
	% Changes the page style to plain - only display page number - no headers
	\pagestyle{plain}
	\tableofcontents
	
	% then clears the page, and goes back to arabic numerals
	\newpage
	\pagestyle{fancy} % sets the page style back to nice headers
	\pagenumbering{arabic}
	\setcounter{page}{1}
	
	%-------------Sections Begin--------------------
	\renewcommand{\thesection}{} % don't show a part number in the ToC
	\titleformat{\section}[hang]{\Large\bfseries}{}{0ex}{}{} % No Part Specifier in title
		\setcounter{section}{0}
	\section{Summary}

\section{Home Screen}



\section{ User}

\subsection{Create New User}

\subsection{Logging in Exisiting user}

\subsection{Logging out user}

\subsection{Unregestered User unable to log in}

\subsection{Review from user}

\section{Database}
f. Searching for an item that exists in the database;
\subsection{Item within the database}
g. Searching for an item that does not exist in the database;
\subsection{Item not within the database}

\subsection{Attempted SQL Injection}
h. Accessing an individual item page;
\section{Pages}
\subsection{Individual Item page}
i. Attempting to use a cross site scripting attack but not being successful;

\section{Multiple resolutions}
	
	%include a summary of the conclusion here
	
	% change the formatting to something a little prettier, and reset the section numbering to 0	

	% toc can go back to numbers now
	\renewcommand{\thesection}{\arabic{section}}
	% show the part specifier
	\titleformat{\section}[hang]{\Large\bfseries}{}{0ex}{{\normalfont Part \thesection}\hspace{.5ex}\textbar\hspace{.5ex}}{}
	
	
	
	\printbibliography
	
	\newpage
	\begin{appendices}
		\section{Brute Force Psudocode}
		\begin{figure}[h]
			\begin{lstlisting}
			ALGORITHM BruteForceMedian(A[0..n-1])
			k <- |n 2|
			for i in 0 to n-1 do
			numsmaller <- 0
			numequal <- 0 
			for j in 0 to n-1 do
			if A[j] < A[i] then
			numsmaller <- numsmaller + 1
			else
			if A[j] = A[i] then
			numequal <- numequal + 1
			if numsmaller < k and k <= (numsmaller + numequal) then
			return A[i]
			\end{lstlisting}
			\caption{\label{Brute Force Psudocode} }
		\end{figure}
		
		\newpage
		
		
		
		\section{Median Psudocode}
		\begin{figure}[h]
			\begin{lstlisting}
			ALGORITHM Median(A[0..n-1])
			if n = 1 then
			return A[0]
			else
			Select(A, 0, |_n/2_|, n-1)
			\end{lstlisting}
			\caption{\label{Median Psudocode} }
		\end{figure}
		
		
		\section{Select Psudocode}
		\begin{figure}[h]
			\begin{lstlisting}
			ALGORITHM Select(A[0..n-1])
			pos <- Partition(A,l,h)
			if pos = m then
			return A[pos]
			if pos > m then
			return Select(A, l, m, pos - 1)
			if pos < m then
			return Select(A, pos + 1, m, h)
			\end{lstlisting}
			\caption{\label{Select Psudocode} }
		\end{figure}
		\newpage
		
		\section{Partition Psudocode}
		\begin{figure}[h]
			\begin{lstlisting}
			ALGORITHM Partition(A[0..n-1])
			for j in l + 1 to h do
			if A[j] < pivotal then
			pivotloc <- pivotloc + 1
			swap(A[pivotloc], A[j])
			swap(A[l], A[pivotloc])
			return pivotloc
			\end{lstlisting}
			\caption{\label{PartitionPsudocode} Partitions array slice A[l..h] by moving element A[l] to the position it would have if the array slice was sorted, and by moving all values in the slice smaller than A[l] to earlier positions, and all values larger than or equal to A[l] to later positions. Returns the index at which  the ‘pivot’ element formerly at location A[l] is placed.} 
			% not sure if we can just include what was said in the actual assignment without referencing, will ask Anthoney when I see him next
		\end{figure}
		
		\newpage
		
		\section{Brute Force Implementation - Basic Operations}
		\begin{figure}[H]
			\begin{lstlisting}
			int BruteForceMedianOps(vector<int> unsortedArray) {
			int lengthOfArray = (int) unsortedArray.size();
			int k = lengthOfArray / 2;
			basicOperations = 0;
			for (int i = 0; i < lengthOfArray - 1; ++i) {
			int numSmaller = 0;
			int numEqual = 0;
			for (int j = 0; j < lengthOfArray - 1; ++j) {
			basicOperations++;
			if (unsortedArray[j] < unsortedArray[i]) {
			numSmaller++;
			} else {
			if (unsortedArray[j] == unsortedArray[i]) {
			numEqual++;
			}
			}
			}
			if (numSmaller < k && k <= (numSmaller + numEqual)) {
			return unsortedArray[i];
			}
			}
			return -1;
			}
			\end{lstlisting}
			\caption{\label{BruteForceImplementationOps} The Brute Force Median finding algorithm which includes finding the amount of basic operations initialised on line 4 and incremented on line 9.}
		\end{figure}
		
		\newpage
		
		\section{Brute Force Implementation - Timed}
		\begin{figure}[H]
			\begin{lstlisting}
			int BruteForceMedianTime(vector<int> unsortedArray) {
			int lengthOfArray = (int) unsortedArray.size();
			double k = ceil(lengthOfArray / 2.0);
			auto start = Clock::now();
			for (int i = 0; i < lengthOfArray - 1; ++i) {
			int numSmaller = 0;
			int numEqual = 0;
			for (int j = 0; j < lengthOfArray - 1; ++j) {
			if (unsortedArray[j] < unsortedArray[i]) {
			numSmaller++;
			} else {
			if (unsortedArray[j] == unsortedArray[i]) {
			numEqual++;
			}
			}
			}
			if (numSmaller < k && k <= (numSmaller + numEqual)) {
			auto end = Clock::now();
			runtime = end - start;
			return unsortedArray[i];
			}
			}
			return -1;
			}
			\end{lstlisting}
			\caption{\label{BruteForceImplementationTimed} This function uses the C++ languages standard library clock function  on line 4 to begin timing a majority of the Brute Force Algorithm. The clock is than stopped at line 18 after the median has been located, fully encaptulating time to perform the algorithm from any other process.}
		\end{figure}
		
		\newpage
		
		\section{Partition Implemenation - Basic Operations}
		\begin{figure}[H]
			\begin{lstlisting}
			int partition_count(vector<int> &array, int l, int h){
			int temp;
			int pivot_value = array[l];
			int pivot_location = l;
			for (int j = l + 1; j < h; j++){
			basicOperations++;
			if (array[j] < pivot_value){
			pivot_location++;
			temp = array[j];
			array[j] = array[pivot_location];
			array[pivot_location] = temp;
			}
			}
			temp = array[pivot_location];
			array[pivot_location] = array[l];
			array[l] = temp;
			return pivot_location;
			}
			\end{lstlisting}
			\caption{\label{PartitionImplementationOps} } 
		\end{figure}
		
		\newpage
		
		\section{Select Implementation - Basic Operations}
		\begin{figure}[H]
			\begin{lstlisting}
			int select_count(vector<int> &array, int l, int m, int h){
			int pos = partition_count(array, l, h);
			if (pos == m){
			return array[pos];
			}
			if (pos > m){
			return select_count(array, l, m, pos - 1);
			}
			return select_count(array, pos + 1, m, h);
			}
			\end{lstlisting}
			\caption{\label{SelectImplemenationOps} }
		\end{figure}
		\newpage
		
		\section{Median Implementation - Basic Operations}
		\begin{figure}[H]
			\begin{lstlisting}
			int selection_median_count(vector<int> &array){
			if (array.size() == 1){
			return array[0];
			}
			return select_count(array, 0, (int)array.size()/2, (int)array.size() - 1);
			}
			\end{lstlisting}
			\caption{\label{MedianImplementationOps} }
		\end{figure}
		
		\newpage
		
		\section{Brute Force Average Efficiency - Basic Operation}
		\label{bruteops}
		We can show that the count of the operations in the average case will be:
		\begin{equation*}
		\ceil*{\frac{n^2}{4}}
		\end{equation*}
		where n is the size of the array.
		For each element less than the median, a full n loops must complete. This is looping through n elements. In the average case, the median element must exist in the middle of the array, and the algorithm stops once it has found the median. 
		The middle of the array is at $\frac{n}{2}$  if there are n elements looped through, until $\frac{n}{4}$ loops have completed, that is $\frac{n}{2}$ repeats n times, or $\frac{n}{2} n * n = {\frac{n^2}{4}} $. Unfortunatley, this does not hold completely true. In cases where n is odd the count of operations is:
		\begin{equation*}
		\ceil*{\frac{n^2}{4}}
		\end{equation*}
		
	\end{appendices}
\end{document}

